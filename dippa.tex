% Lines starting with a percent sign (%) are comments. LaTeX will
% not process those lines. Similarly, everything after a percent
% sign in a line is considered a comment. To produce a percent sign
% in the output, write \% (backslash followed by the percent sign).
% ==================================================================
% Usage instructions:
% ------------------------------------------------------------------
% The file is heavily commented so that you know what the various
% commands do. Feel free to remove any comments you don't need from
% your own copy. When redistributing the example thesis file, please
% retain all the comments for the benefit of other thesis writers!
% ==================================================================
% Compilation instructions: 
% ------------------------------------------------------------------
% Use pdflatex to compile! Input images are expected as PDF files.
% Example compilation:
% ------------------------------------------------------------------
% > pdflatex thesis-example.tex
% > bibtex thesis-example
% > pdflatex thesis-example.tex
% > pdflatex thesis-example.tex
% ------------------------------------------------------------------
% You need to run pdflatex multiple times so that all the cross-references
% are fixed. pdflatex will tell you if you need to re-run it (a warning
% will be issued)
% ------------------------------------------------------------------
% Compilation has been tested to work in ukk.cs.hut.fi and kosh.hut.fi
% - if you have problems of missing .sty -files, then the local LaTeX
% environment does not have all the required packages installed.
% For example, when compiling in vipunen.hut.fi, you get an error that
% tikz.sty is missing - in this case you must either compile somewhere
% else, or you cannot use TikZ graphics in your thesis and must therefore
% remove or comment out the tikz package and all the tikz definitions.
% ------------------------------------------------------------------

% General information
% ==================================================================
% Package documentation:
%
% The comments often refer to package documentation. (Almost) all LaTeX
% packages have documentation accompanying them, so you can read the
% package documentation for further information. When a package 'xxx' is
% installed to your local LaTeX environment (the document compiles
% when you have \usepackage{xxx} and LaTeX does not complain), you can
% find the documentation somewhere in the local LaTeX texmf directory
% hierarchy. In ukk.cs.hut.fi, this is /usr/texlive/2008/texmf-dist,
% and the documentation for the titlesec package (for example) can be
% found at /usr/texlive/2008/texmf-dist/doc/latex/titlesec/titlesec.pdf.
% Most often the documentation is located as a PDF file in
% /usr/texlive/2008/texmf-dist/doc/latex/xxx, where xxx is the package name;
% however, documentation for TikZ is in
% /usr/texlive/2008/texmf-dist/doc/latex/generic/pgf/pgfmanual.pdf
% (this is because TikZ is a front-end for PGF, which is meant to be a
% generic portable graphics format for LaTeX).
% You can try to look for the package manual using the ``find'' shell
% command in Linux machines; the find databases are up-to-date at least
% in ukk.cs.hut.fi. Just type ``find xxx'', where xxx is the package
% name, and you should find a documentation file.
% Note that in some packages, the documentation is in the DVI file
% format. In this case, you can copy the DVI file to your home directory,
% and convert it to PDF with the dvipdfm command (or you can read the
% DVI file directly with a DVI viewer).
%
% If you can't find the documentation for a package, just try Googling
% for ``latex packagename''; most often you can get a direct link to the
% package manual in PDF format.
% ------------------------------------------------------------------


% Document class for the thesis is report
% ------------------------------------------------------------------
% You can change this but do so at your own risk - it may break other things.
% Note that the option pdftext is used for pdflatex; there is no
% pdflatex option.
% ------------------------------------------------------------------
\documentclass[12pt,a4paper,oneside,pdftex]{report}

% The input files (tex files) are encoded with the latin-1 encoding
% (ISO-8859-1 works). Change the latin1-option if you use UTF8
% (at some point LaTeX did not work with UTF8, but I'm not sure
% what the current situation is)
\usepackage[utf8]{inputenc}
% OT1 font encoding seems to work better than T1. Check the rendered
% PDF file to see if the fonts are encoded properly as vectors (instead
% of rendered bitmaps). You can do this by zooming very close to any letter
% - if the letter is shown pixelated, you should change this setting
% (try commenting out the entire line, for example).
\usepackage[OT1]{fontenc}
% The babel package provides hyphenating instructions for LaTeX. Give
% the languages you wish to use in your thesis as options to the babel
% package (as shown below). You can remove any language you are not
% going to use.
% Examples of valid language codes: english (or USenglish), british,
% finnish, swedish; and so on.
\usepackage[finnish,english]{babel}


% Font selection
% ------------------------------------------------------------------
% The default LaTeX font is a very good font for rendering your
% thesis. It is a very professional font, which will always be
% accepted.
% If you, however, wish to spicen up your thesis, you can try out
% these font variants by uncommenting one of the following lines
% (or by finding another font package). The fonts shown here are
% all fonts that you could use in your thesis (not too silly).
% Changing the font causes the layouts to shift a bit; you many
% need to manually adjust some layouts. Check the warning messages
% LaTeX gives you.
% ------------------------------------------------------------------
% To find another font, check out the font catalogue from
% http://www.tug.dk/FontCatalogue/mathfonts.html
% This link points to the list of fonts that support maths, but
% that's a fairly important point for master's theses.
% ------------------------------------------------------------------
% <rant>
% Remember, there is no excuse to use Comic Sans, ever, in any
% situation! (Well, maybe in speech bubbles in comics, but there
% are better options for those too)
% </rant>

% \usepackage{palatino}
\usepackage{tgpagella}



% Optional packages
% ------------------------------------------------------------------
% Select those packages that you need for your thesis. You may delete
% or comment the rest.

% Natbib allows you to select the format of the bibliography references.
% The first example uses numbered citations:
%\usepackage[square,sort&compress,numbers]{natbib}
% The second example uses author-year citations.
% If you use author-year citations, change the bibliography style (below);
% acm style does not work with author-year citations.
% Also, you should use \citet (cite in text) when you wish to refer
% to the author directly (\citet{blaablaa} said blaa blaa), and
% \citep when you wish to refer similarly than with numbered citations
% (It has been said that blaa blaa~\citep{blaablaa}).
\usepackage[square]{natbib}

% The alltt package provides an all-teletype environment that acts
% like verbatim but you can use LaTeX commands in it. Uncomment if
% you want to use this environment.
% \usepackage{alltt}

% The eurosym package provides a euro symbol. Use with \euro{}
\usepackage{eurosym}

% Verbatim provides a standard teletype environment that renderes
% the text exactly as written in the tex file. Useful for code
% snippets (although you can also use the listings package to get
% automatic code formatting).
\usepackage{verbatim}

% The listing package provides automatic code formatting utilities
% so that you can copy-paste code examples and have them rendered
% nicely. See the package documentation for details.
% \usepackage{listings}

% The fancuvrb package provides fancier verbatim environments
% (you can, for example, put borders around the verbatim text area
% and so on). See package for details.
% \usepackage{fancyvrb}

% Supertabular provides a tabular environment that can span multiple
% pages.
%\usepackage{supertabular}
% Longtable provides a tabular environment that can span multiple
% pages. This is used in the example acronyms file.
\usepackage{longtable}

% The fancyhdr package allows you to set your the page headers
% manually, and allows you to add separator lines and so on.
% Check the package documentation.
% \usepackage{fancyhdr}

% Subfigure package allows you to use subfigures (i.e. many subfigures
% within one figure environment). These can have different labels and
% they are numbered automatically. Check the package documentation.
\usepackage{subfigure}

% The titlesec package can be used to alter the look of the titles
% of sections, chapters, and so on. This example uses the ``medium''
% package option which sets the titles to a medium size, making them
% a bit smaller than what is the default. You can fine-tune the
% title fonts and sizes by using the package options. See the package
% documentation.
\usepackage[medium]{titlesec}

% The TikZ package allows you to create professional technical figures.
% The learning curve is quite steep, but it is definitely worth it if
% you wish to have really good-looking technical figures.
\usepackage{tikz}
% You also need to specify which TikZ libraries you use
\usetikzlibrary{positioning}
\usetikzlibrary{calc}
\usetikzlibrary{arrows}
\usetikzlibrary{decorations.pathmorphing,decorations.markings}
\usetikzlibrary{shapes}
\usetikzlibrary{patterns}

\renewcommand{\ATDEGREEPROG}{Information Networks}%
%\renewcommand*{\school}{School of Science}%
%\renewcommand*{\korkeakoulu}{Perustieteiden korkeakoulu}%
%\tutkohj{Informaatioverkostot}%
%\degreeprogram{Information Networks}% 
%\renewcommand*{\absheadfin}{diplomity\"on tiivistelm\"a}
% The aalto-thesis package provides typesetting instructions for the
% standard master's thesis parts (abstracts, front page, and so on)
% Load this package second-to-last, just before the hyperref package.
% Options that you can use:
%   mydraft - renders the thesis in draft mode.
%             Do not use for the final version.
%   doublenumbering - [optional] number the first pages of the thesis
%                     with roman numerals (i, ii, iii, ...); and start
%                     arabic numbering (1, 2, 3, ...) only on the
%                     first page of the first chapter
%   twoinstructors  - changes the title of instructors to plural form
%   twosupervisors  - changes the title of supervisors to plural form
%\usepackage[mydraft,twosupervisors]{aalto-thesis}
\usepackage[mydraft,doublenumbering]{aalto-thesis}
%\usepackage{aalto-thesis}


% Hyperref
% ------------------------------------------------------------------
% Hyperref creates links from URLs, for references, and creates a
% TOC in the PDF file.
% This package must be the last one you include, because it has
% compatibility issues with many other packages and it fixes
% those issues when it is loaded.
\RequirePackage[pdftex]{hyperref}
% Setup hyperref so that links are clickable but do not look
% different
\hypersetup{colorlinks=false,raiselinks=false,breaklinks=true}
\hypersetup{pdfborder={0 0 0}}
\hypersetup{bookmarksnumbered=true}
% The following line suggests the PDF reader that it should show the
% first level of bookmarks opened in the hierarchical bookmark view.
\hypersetup{bookmarksopen=true,bookmarksopenlevel=1}
% Hyperref can also set up the PDF metadata fields. These are
% set a bit later on, after the thesis setup.


% Thesis setup
% ==================================================================
% Change these to fit your own thesis.
% \COMMAND always refers to the English version;
% \FCOMMAND refers to the Finnish version; and
% \SCOMMAND refers to the Swedish version.
% You may comment/remove those language variants that you do not use
% (but then you must not include the abstracts for that language)
% ------------------------------------------------------------------
% If you do not find the command for a text that is shown in the cover page or
% in the abstract texts, check the aalto-thesis.sty file and locate the text
% from there.
% All the texts are configured in language-specific blocks (lots of commands
% that look like this: \renewcommand{\ATCITY}{Espoo}.
% You can just fix the texts there. Just remember to check all the language
% variants you use (they are all there in the same place).
% ------------------------------------------------------------------
\newcommand{\TITLE}{Service dominant logic in successful \\public procurement}
\newcommand{\FTITLE}{Onnistuneen tiedonjakamisen elementit pk-palveluyritysten ja julkisten hankkijoiden välillä}

\newcommand{\SUBTITLE}{Method for fulfilling user needs}
\newcommand{\FSUBTITLE}{Metodi tarveperusteisiin hankintoihin}

\newcommand{\DATE}{June 18, 2013}
\newcommand{\FDATE}{18. kesäkuuta 2013}

% Supervisors and instructors
% ------------------------------------------------------------------
% If you have two supervisors, write both names here, separate them with a
% double-backslash (see below for an example)
% Also remember to add the package option ``twosupervisors'' or
% ``twoinstructors'' to the aalto-thesis package so that the titles are in
% plural.
% Example of one supervisor:
%\newcommand{\SUPERVISOR}{Professor Antti Ylä-Jääski}
%\newcommand{\FSUPERVISOR}{Professori Antti Ylä-Jääski}
%\newcommand{\SSUPERVISOR}{Professor Antti Ylä-Jääski}
% Example of twosupervisors:
\newcommand{\SUPERVISOR}{Professor Riitta Smeds, D.Sc. (Tech.)}
\newcommand{\FSUPERVISOR}{Professori Riitta Smeds, TkT}

% If you have only one instructor, just write one name here
\newcommand{\INSTRUCTOR}{Soile Pohjonen, LL.D.}
\newcommand{\FINSTRUCTOR}{Soile Pohjonen, OTT}

% If you have two instructors, separate them with \\ to create linefeeds
% \newcommand{\INSTRUCTOR}{Olli Ohjaaja M.Sc. (Tech.)\\
%  Elli Opas M.Sc. (Tech)}
%\newcommand{\FINSTRUCTOR}{Diplomi-insinööri Olli Ohjaaja\\
%  Diplomi-insinööri Elli Opas}
%\newcommand{\SINSTRUCTOR}{Diplomingenjör Olli Ohjaaja\\
%  Diplomingenjör Elli Opas}

% If you have two supervisors, it is common to write the schools
% of the supervisors in the cover page. If the following command is defined,
% then the supervisor names shown here are printed in the cover page. Otherwise,
% the supervisor names defined above are used.
% \newcommand{\COVERSUPERVISOR}{Professor Riitta Smeds}

% The same option is for the instructors, if you have multiple instructors.
% \newcommand{\COVERINSTRUCTOR}{Olli Ohjaaja M.Sc. (Tech.), Aalto University\\
%  Elli Opas M.Sc. (Tech), Aalto SCI}


% Other stuff
% ------------------------------------------------------------------
\newcommand{\PROFESSORSHIP}{Business processes and services in digital networks}
\newcommand{\FPROFESSORSHIP}{Liiketoiminta- ja palveluprosessit tietoverkostoissa}

% Professorship code is the same in all languages
\newcommand{\PROFCODE}{TU-124}
\newcommand{\KEYWORDS}{public procurement, SMEs, co-creation}
\newcommand{\FKEYWORDS}{avainsanat suomeksi}

\newcommand{\LANGUAGE}{English}
\newcommand{\FLANGUAGE}{Englanti}


% Author is the same for all languages
\newcommand{\AUTHOR}{Asta Länsimies}


% Currently the English versions are used for the PDF file metadata
% Set the PDF title
\hypersetup{pdftitle={\TITLE\ \SUBTITLE}}
% Set the PDF author
\hypersetup{pdfauthor={\AUTHOR}}
% Set the PDF keywords
\hypersetup{pdfkeywords={\KEYWORDS}}
% Set the PDF subject
\hypersetup{pdfsubject={Master's Thesis}}


% Layout settings
% ------------------------------------------------------------------

% When you write in English, you should use the standard LaTeX
% paragraph formatting: paragraphs are indented, and there is no
% space between paragraphs.
% When writing in Finnish, we often use no indentation in the
% beginning of the paragraph, and there is some space between the
% paragraphs.

% If you write your thesis Finnish, uncomment these lines; if
% you write in English, leave these lines commented!
% \setlength{\parindent}{0pt}
% \setlength{\parskip}{1ex}

% Use this to control how much space there is between each line of text.
% 1 is normal (no extra space), 1.3 is about one-half more space, and
% 1.6 is about double line spacing.
% \linespread{1} % This is the default
\linespread{1.3}

% Bibliography style
% acm style gives you a basic reference style. It works only with numbered
% references.
%\bibliographystyle{acm}
% Plainnat is a plain style that works with both numbered and name citations.
\bibliographystyle{plainnat}


% Extra hyphenation settings
% ------------------------------------------------------------------
% You can list here all the files that are not hyphenated correctly.
% You can provide many \hyphenation commands and/or separate each word
% with a space inside a single command. Put hyphens in the places where
% a word can be hyphenated.
% Note that (by default) LaTeX will not hyphenate words that already
% have a hyphen in them (for example, if you write ``structure-modification
% operation'', the word structure-modification will never be hyphenated).
% You need a special package to hyphenate those words.
\hyphenation{di-gi-taa-li-sta yksi-suun-tai-sta}



% The preamble ends here, and the document begins.
% Place all formatting commands and such before this line.
% ------------------------------------------------------------------
\begin{document}
% This command adds a PDF bookmark to the cover page. You may leave
% it out if you don't like it...
\pdfbookmark[0]{Cover page}{bookmark.0.cover}
% This command is defined in aalto-thesis.sty. It controls the page
% numbering based on whether the doublenumbering option is specified
\startcoverpage

% Cover page
% ------------------------------------------------------------------
% Options: finnish, english, and swedish
% These control in which language the cover-page information is shown
\coverpage{english}


% Abstracts
% ------------------------------------------------------------------
% Include an abstract in the language that the thesis is written in,
% and if your native language is Finnish or Swedish, one in that language.

% Abstract in English
% ------------------------------------------------------------------
\thesisabstract{english}{


Fixme is a command that helps you identify parts of your thesis that still
require some work.
\fixme{Abstract text goes here (and this is an example how to use fixme).} When compiled in the custom \texttt{mydraft} mode, text
parts tagged with fixmes are shown in bold and with fixme tags around them. When
compiled in normal mode, the fixme-tagged text is shown normally (without
special formatting). The draft mode also causes the ``Draft'' text to appear on
the front page, alongside with the document compilation date. The custom
\texttt{mydraft} mode is selected by the \texttt{mydraft} option given for the
package \texttt{aalto-thesis}, near the top of the \texttt{thesis-example.tex}
file.

}
% Abstract in Finnish
% ------------------------------------------------------------------
\thesisabstract{finnish}{
But I must explain to you how all this mistaken idea of denouncing pleasure and praising pain was born 
.
}
% Abstract in Swedish
% ------------------------------------------------------------------
% \thesisabstract{swedish}


% Acknowledgements
% ------------------------------------------------------------------
% Select the language you use in your acknowledgements
\selectlanguage{english}

% Uncomment this line if you wish acknoledgements to appear in the
% table of contents
%\addcontentsline{toc}{chapter}{Acknowledgements}

% The star means that the chapter isn't numbered and does not
% show up in the TOC
\chapter*{Acknowledgements}

I wish to thank all students who use \LaTeX\ for formatting their theses,
because theses formatted with \LaTeX\ are just so nice.

Thank you, and keep up the good work!
\vskip 10mm

\noindent Espoo, \DATE
\vskip 5mm
\noindent\AUTHOR

% Acronyms
% ------------------------------------------------------------------
% Use \cleardoublepage so that IF two-sided printing is used
% (which is not often for masters theses), then the pages will still
% start correctly on the right-hand side.
\cleardoublepage
% Example acronyms are placed in a separate file, acronyms.tex
% \input{acronyms}

\addcontentsline{toc}{chapter}{Abbreviations and Acronyms}
\chapter*{Abbreviations and Acronyms}

% The longtable environment should break the table properly to multiple pages,
% if needed

\noindent
\begin{longtable}{@{}p{0.25\textwidth}p{0.7\textwidth}@{}}
HILMA & The official electronic web-based procurement notification channel, administered by Ministry of Employment and Economy in Finland, in which procuring entities publish their contract notices that exceed the national 
thresholds \\
e.g.& for example (do not list here this kind of common acronymbs or abbreviations, but only those that are essential for understanding the content of your thesis. \\
note & Note also, that this list is not compulsory, and should be omitted if you have only few abbreviations

\end{longtable}


% Table of contents
% ------------------------------------------------------------------
\cleardoublepage
% This command adds a PDF bookmark that links to the contents.
% You can use \addcontentsline{} as well, but that also adds contents
% entry to the table of contents, which is kind of redundant.
% The text ``Contents'' is shown in the PDF bookmark.
\pdfbookmark[0]{Contents}{bookmark.0.contents}
\tableofcontents

% List of tables
% ------------------------------------------------------------------
% You only need a list of tables for your thesis if you have very
% many tables. If you do, uncomment the following two lines.
% \cleardoublepage
% \listoftables

% Table of figures
% ------------------------------------------------------------------
% You only need a list of figures for your thesis if you have very
% many figures. If you do, uncomment the following two lines.
% \cleardoublepage
% \listoffigures

% The following label is used for counting the prelude pages
\label{pages-prelude}
\cleardoublepage

%%%%%%%%%%%%%%%%% The main content starts here %%%%%%%%%%%%%%%%%%%%%
% ------------------------------------------------------------------
% This command is defined in aalto-thesis.sty. It controls the page
% numbering based on whether the doublenumbering option is specified
\startfirstchapter

% Add headings to pages (the chapter title is shown)
\pagestyle{headings}

% The contents of the thesis are separated to their own files.
% Edit the content in these files, rename them as necessary.
% ------------------------------------------------------------------

% \input{1introduction.tex}

\chapter{Introduction}
\label{chapter:intro}
\label{Background}

EU wants to increase the procurements from SME’s. SME’s as public sector service providers help them to grow their business also in the private sector and in the end benefit the whole economy. (lähde?)

The Finnish Act of Public Procurements requires a formal tender process. The  act  is  applied  when  the  procurement  exceeds  certain  threshold  values. In most cases, public procurements exceeding 30 000 euros in value are subject to the Public Procurement law and must be tendered in a national level. The procuring entity must set the requirements for the contract notice and the invitation to tender prior to receiving any offers, and the tender cannot be changed afterwards in order to ensure non-discriminating and unbiased procurement process for all tenders. ~\citep{Hankintaopas}

Because the requirements cannot be changed during the tendering process, public procurer needs to know exactly what to ask for. The companies participating in tendering cannot offer anything conflicting with or additional to the invitation to tender. The responsibility of defining user needs and requirements is with the procuring entity completely. The information companies have about the service offering on the market is not passed effectively on to the procuring party. The procurer may not be aware of different solutions available. Companies may be experienced in fulfilling customer needs and have knowledge of best solutions available. (lähde?)

The public sector should be made more efficient. Defining end user needs is critical.

The supplier candidates do not know how to make offers correctly – companies are not aware of the public procurement process and procurement law. 

The problem is that services are procured using a product dominant logic instead of a service dominant logic – how to transform the procurement logic?

The goal of this study is to investigate, how companies and public procurers can engage in a dialogue before the tendering process – to provide the procurer with better knowledge of solutions available, and the companies of the tendering process.  Companies that participate or are interested in participating in tendering, can notify the procurer of the different options for products and services they can offer only before the tendering process begins.
How the public procurers can transform into a service dominant logic?
\section{Background}

\subsection{SME's in public procurement}
In this chapter I will introduce to you the SME's role in public procurement. 

\subsection{VISO project}
Here I introduce the research project VISO.

\section{Research questions}
\begin{itemize}
% You can use this command to set the items in the list closer to each other
% (ITEM SEParation, the vertical space between the list  items)
\setlength{\itemsep}{0pt}
\item How is service dominant logic used in the current public service procurement process?
\item How can the public procurer increase market dialogue with service suppliers?
\end{itemize}


A wonderful serenity has taken possession of my entire soul, like these sweet mornings of spring which I enjoy with my whole heart. I am alone, and feel the charm of existence in this spot, which was created for the bliss of souls like mine. I am so happy, my dear friend, so absorbed in the exquisite sense of mere tranquil existence, that I neglect my talents. I should be incapable of drawing a single stroke at the present moment; and yet I feel that I never was a greater artist than now. When, while the lovely valley teems with vapour around me, and the meridian sun strikes the upper surface of the impenetrable foliage of my trees, and but a few stray gleams steal into the inner sanctuary, I throw myself down among the tall grass by the trickling stream; and, as I lie close to the earth, a thousand unknown plants are noticed by me: when I hear the buzz of the little world among the stalks, and grow familiar with the countless indescribable forms of the insects and flies, then I feel the presence of the Almighty, who formed us in his own image, and the breath.

\section{Research process}

Far far away, behind the word mountains, far from the countries Vokalia and Consonantia, there live the blind texts. Separated they live in Bookmarksgrove right at the coast of the Semantics, a large language ocean. A small river named Duden flows by their place and supplies it with the necessary regelialia. It is a paradisematic country, in which roasted parts of sentences fly into your mouth. Even the all-powerful Pointing has no control about the blind texts it is an almost unorthographic life One day however a small line of blind text by the name of Lorem Ipsum decided to leave for the far World of Grammar. The Big Oxmox advised her not to do so, because there were thousands of bad Commas, wild Question Marks and devious Semikoli, but the Little Blind Text didn’t listen. She packed her seven versalia, put her initial into the belt and made herself on the way. When she reached the first hills of the Italic Mountains, she had a last view back on the skyline of her hometown Bookmarksgrove, the headline of Alphabet Village and the subline of her own road, the Line Lane.

\section{Research methods}

One morning, when Gregor Samsa woke from troubled dreams, he found himself transformed in his bed into a horrible vermin. He lay on his armour-like back, and if he lifted his head a little he could see his brown belly, slightly domed and divided by arches into stiff sections. The bedding was hardly able to cover it and seemed ready to slide off any moment. His many legs, pitifully thin compared with the size of the rest of him, waved about helplessly as he looked. "What's happened to me?" he thought. It wasn't a dream. His room, a proper human room although a little too small, lay peacefully between its four familiar walls. A collection of textile samples lay spread out on the table - Samsa was a travelling salesman - and above it there hung a picture that he had recently cut out of an illustrated magazine and housed in a nice, gilded frame. It showed a lady fitted out with a fur hat and fur boa who sat upright, raising a heavy fur muff that covered the whole of her lower arm towards the viewer. Gregor then turned to look out the window at the dull weather.

\subsection{Case study}
A collection of textile samples lay spread out on the table - Samsa was a travelling salesman - and above it there hung a picture that he had recently cut out of an illustrated magazine and housed in a nice, gilded frame. It showed a lady fitted out with a fur hat and fur boa who sat upright, raising a heavy fur muff that covered the whole of her lower arm towards the viewer. Gregor then turned to look out the window at the dull weather.

\subsection{Literature review}
He lay on his armour-like back, and if he lifted his head a little he could see his brown belly, slightly domed and divided by arches into stiff sections. The bedding was hardly able to cover it and seemed ready to slide off any moment.

\subsection{Interviews and simulation day}
One morning, when Gregor Samsa woke from troubled dreams, he found himself transformed in his bed into a horrible vermin. 

\subsection{Workshop}
Gregor then turned to look out the window at the dull weather.



I have read the information from the university master's thesis
pages~\cite{ThesisInstructions} before starting the thesis.  You
should also go through the thesis grading
instructions~\cite{ThesisGrading} together with your instructor and/or
supervisor in the beginning of your work.

\section{Structure of the Thesis}
\label{section:structure}

This thesis consists on five parts: first, the theoretical background of public procurement will be introduced. Then, the elements of successful pre-procurement co-creation and knowledge sharing will be discussed based on the empirical data. Suggestions for methods and practices will be introduced and results from testing are brought to attention. Finally, the implications for future research are discussed.

%Contents
%Abstract
%1.    Introduction
% 1.1.	Background
%  1.1.1.	SME’s in public procurement
%  1.1.2.	VISO
% 1.2.	Research questions
% 1.3.	Research process
% 1.4.	Research methods
%  1.4.1.	 Case study
%  1.4.2.	Literature review
%  1.4.3.	 Interviews and simulation day
%  1.4.4.	 Workshop
% 1.5.	Structure of the Thesis
%2.	Theoretical background
% 2.1.	Public procurement
%  2.1.1.	 Public procurement Act and the public tendering process
%  2.1.2.	 Small and medium sized companies in public procurement tendering
%  2.1.3.	 
% 2.2.	Service procurement
% 2.3.	Service dominant logic
%  2.3.1.	 Service dominant logic in procurement
%  2.3.2.	 Service dominant logic versus product dominant logic
% 2.4.	Knowledge transfer methods in procurement
% 2.5.	Theoretical framework
%3.	Empirical study
%4.	Discussion
%5.	Conclusions
%6.	Implications
%References


% \input{2background.tex}

\chapter{Theoretical background}
\label{chapter:theory}

In this chapter, the theoretical background to the research is introduced and discussed. First, the principles of public service procurement and the tendering process are described. The role of SME’s in public procurement is discussed. Then, I will shortly introduce the principles of service procurement. In the third section, service dominant logic and its relation to product dominant logic are discussed. In the final section of this chapter, the prior research on knowledge transfer methods in procurement is reviewed and discussed. Finally, the theoretical background to the research is summed up and built into a theoretical framework.

\section{Public procurement}
\subsection {Principles of Public Procurement}

what is public procurement? (Karinkanta et al. 2012)

Public procurement accounts for XX  of GDP in Finland (EU?). Thus, public procurement has a significant role in the Finnish economy. 

In Finland during 2006 to 2008, share of services in the above-threshold procurement value was 34.

Considerable share of public contracts are made by municipalities. In 2008, local authorities accounted for around 25 of public contracts above EU thresholds, which is relative majority (European Commission 2010). On a national level there is no statistical information on the share of municipalities, but it is considered significant.

\subsection {The Finnish Act on Public Contracts and tendering process}

The threshold values of public contracts are defined to 30 000 euros on a national level. Contracts that are smaller in value than the threshold can be made without following the competitive tendering process. The threshold value for works is 150 000 euros on a national level (Karinkanta et al. 2012)
[Table of threshold values]
There are different tendering procedures that can be used in public procurement, of which the open procedure tendering should be used primarily. There are other procedures that can be used in specific situations, such as restricted procedure and negotiated procedure, but they will not be in the focus of this study. 

from the Online
Writing Lab
(OWL)\footnote{http://owl.english.purdue.edu/owl/resource/574/02/} of
Purdue University or Strunk's Elements of
Style\footnote{http://www.bartleby.com/141/}. Remember that footnotes
are additional information, and they are seldom used.  If you refer to a source, you do no
not use footnote. The right command for the references is \emph{cite}.



\section{Service procurement}

Never ever copy anything into your theses from somebody else's text
(nor your own previously published text). Never. Not even for starting
point to be rewritten later. The risk is that you forgot the copied
text to your thesis and end up to be accused of plagiarism. Plagiarism
is a serious crime in studies and science and can ruin your career
even its beginning. To repeate: never cut and paste text into your
thesis!

\subsection{Jaadijaa}

All work is based on someone else's work. You should find the relevant
sources of your field and choose the best of them. Also, you should
refer to the original source where a fact has been mentioned first
time. Remember source evaluation (criticism) with all sources you
find.

Good starting points for finding references in computer science are:
\begin{itemize}
% You can use this command to set the items in the list closer to each other
% (ITEM SEParation, the vertical space between the list items)
\setlength{\itemsep}{0pt}
\item Nelli Portal (Aalto Library): \url{http://www.nelliportaali.fi}
\item ACM Digital library: \url{http://portal.acm.org/}
\item IEEExplore: \url{http://ieeexplore.ieee.org}
\item ScienceDirect: \url{http://www.sciencedirect.com/}
\item \ldots although Google Scholar (\url{http://scholar.google.com/}) will
find links to most of the articles from the abovementioned sources, if you
search from within the university network
\end{itemize}

Some of the publishers do not offer all the text of the articles
freely, but the library has agreed on the rights to use the whole
text. Thus, you should sometimes use computers in the domain of the
university in order to get the full text. Sometimes the Nelli Portal
can also help getting the whole article instead of just the abstract.
The library has also brief instrucions how to find
information~\cite{howfindinfo}.

Instead of normal Google, use Google Scholar
(\url{http://scholar.google.fi/}). It finds academic publications whereas
normal Google find too much commercial advertisements or otherwise
biased information. Wikipedia articles should be referred to in the master
thesis only very, very seldomly. You can use Wikipedia for understanding
some basics and finding more sources, but often you cannot be sure if
the article is correct and unbiased.

One important part of the sources that you have found is the reference
list. This way you can find the original sources that all the other
research of the field refer. Often you can also find more information
with the name of the researchers that are often referred in the
articles.

\subsection{Blah}

The main point in referring to sources is to separate your own
thinking and text from that of others. Facts of the research area can
be given without reference, but otherwise you should refer to
sources. This means two things: marking the source in the text where
it has been used, and listing the sources usually in the end of the
thesis in a way that help the reader to find the original source.

There are several bibliography styles, meaning how to form the
bibliography in the end of the thesis. Aalto's library has good
instructions for many styles~\cite{bibinstructions}. You should ask
from your supervisor or instructors which style you should use. This
thesis template uses the number style that is often used in software
engineering. The other style also used in the CS field,
e.g. usability, is the Harvard style where instead of numbers, the
reference is marked into the text with author's name and publishing
year. Other areas use also many other styles for making the lists and
marking the references.

In addition to the list in the end of the thesis, you have to mark the
source in the text where the source is used. There are three places
for the reference: in a sentence before the period, in the end of a
sentence after the period, or in the end of a paragraph. All of them
have different meaning. The main point is that first you paraphrase
the source using your own words and then mark the source. Next, we
give short examples that are marked with \emph{emphasised text}.

\emph{Haapasalo~\cite{HaapasaloThesis} researched database algorithms
  that allows use of previous versions of the content stored in the
  database.} This kind of marking means that this paragraph (or until
the next reference is given) is based on the source mentioned in the
beginning.  Giving the source you should use only the family name of
the first author of the article, and not give any hints about what is
the type of the article that is referred.

\emph{B+-trees offers one way to index data that is stored in to a
  database. Multiversion B+-trees (MVBT) offer also a way to restore
  the data from previous versions of the database. Concurrent MVBT
  allows many simultaneous updates to the database that is was not
  possible with MVBT.~\cite{HaapasaloThesis}} When the marking is
after the period, the reference is retrospective: all the paragraph
(or after previous reference marking) is based on the source given in
its end. If the content is very broad, you can start with saying
\emph{According to Haapasalo}, then continue referring the source with
several separate sentences, and in the end put the marking of your
source \emph{ that shows that CMVBT are the
  best. ~\cite{HaapasaloThesis}}.

If your paragraph has several sources, the above mentioned styles are
not proper. The reader of your thesis cannot know which of your
sources give which of the statements. In this case, it is better to
use more finegraded refering where the reference markings that are
embedded in the sentences. For example, \emph{the multiversion B+-tree
  (MVBT) index of Becker et al.~\cite{becker:1996:mvbt} allows database
  users to query old versions of the database, but the index is not
  transactional.
  It's successor, the transactional MBVT (TMVBT), allows a single transaction
  running in its own thread or process to update the database concurrently
  with other transactions that only read the
  database~\cite{haapasalo:2009:tmvbt}.
  Further development, titled the concurrent MBVT (CMVBT),
  allows several transactions to perform updates to the database at the same
  time~\cite{HaapasaloThesis}}.
  Here, the references are marked before
  the period in the sentences where they are used.

Finally, direct quotes are allowed. However, often you should avoid
them since they do not usually fit in to your text very well. Using
direct quotes has two tricks: quotation marks and the source.  \emph{
  ``Even though deletions in a multiversion index must not physically
  delete the history of the data items, queries and range scans can
  become more efficient, if the leaf pages of the index structure are
  merged to retain optimality.''~\cite{HaapasaloThesis}} Quotes are
hard to make neatly since you should use only as much as needed
without changing the text. Moreover, you often do not really
understand what the author has mentioned with his wordings if you
cannot write the same with your own words. Remember also that never
cut and paste anything without marking the quotation marks right away,
and in general, never cut and paste anything at all!

Sometimes getting the original source can be almost impossible. In an
extremely desperate situation, you can refer with structure \emph{mr
  X~[\ldots] according to ms Y~[\ldots] defined that}, if you find a
source that refers to the original source. Note also that the
reference marking is never used as sentence element (example of how
\textbf{not} to do it: \emph{\cite{HaapasaloThesis} describes
an optimal algorithm for indexing multiversiond databases.}).



% \input{3environment.tex}

\chapter{Empirical study}
\label{chapter:environment}

A problem instance is rarely totally independent of its environment.
Most often you need to describe the environment you work in, what
limits there are and so on. This is a good place to do that. First we
tell you about the LaTeX working environments and then is an example
from an thesis written some years ago.


\chapter{Discussion}
\label{chapter:discussion}

To create \LaTeX\ documents you need two things: a \LaTeX\ environment for
compiling your documents and a text editor for writing them.

\chapter{Conclusions}

Fortunately \LaTeX\ can nowadays be found for any (modern) computer
environment, be it Linux, Windows, or Macintosh.
For Linuxes (and other Unix clones) and Macs, I'd recommend \emph{TeX
Live}~\cite{TeXLive}, which is the current default \LaTeX\ distribution for
many Linux flavors such as Fedora, Debian, Ubuntu, and Gentoo.
TeX Live is the replacement for the older \emph{teTeX}, which is
no longer developed.

TeX Live works also for Windows machines (at least according to their web
site); however, I have used \emph{MiKTeX}~\cite{MiKTeX} and can recommend it
for Windows.
MiKTeX has a nice package manager and automatically fetches missing packages
f.

You can write \LaTeX\ documents with any text editor you like, but having
syntax coloring options and such really helps a lot.
My personal favourite for editing \LaTeX\ is the
\emph{TeXlipse}~\cite{TeXlipse} plugin for the Eclipse IDE~\cite{Eclipse}.
Eclipse is an open-source integrated development environment (IDE) initially
created for writing Java code, but it currently has support for editing
languages such as C, C++, JavaScript, XML, HTML, and many more.
The TeXlipse plugin allows you to edit and compile \LaTeX\ documents directly
in Eclipse, and compilation errors and warnings are shown in the Eclipse
\emph{Problems} dialog so that you can locate and fix the issues easily.
The plugin also supports reference traversal so that you can locate the source
line where a label or a citation is defined.

Eclipse is an entire development environment, so it may feel a bit heavy-weight
for editing a document.
If you are looking for a more light-weight option, check out TeXworks.
TeXworks is a \LaTeX\ editor that is packaged with the newer MiKTeX
distributions, and it can be acquired from \url{http://www.tug.org/texworks/}.

And if you are attached to your \emph{emacs} or \emph{vim} editor, you
can of course edit your \LaTeX\ documents with them.
Emacs at least has syntax coloring and you can compile your document with a key
binding, so this may be a good option if you prefer working with the standard
Linux text editors.

%\section{Graphics}

When you use \texttt{pdflatex} to render your thesis, you can include PDF images
directly, as shown by Figure~\ref{fig:indica_model} below.

\begin{figure}[ht]
  \begin{center}
    \includegraphics[width=\textwidth]{example_indica_model.pdf}
    \caption{The INDICA two-layered value chain model.}
    \label{fig:indica_model}
  \end{center}
\end{figure}

You can also include JPEG or PNG files, as shown by Figure~\ref{fig:eeyore}.

\begin{figure}[ht]
  \begin{center}
    \includegraphics[width=9cm]{example_ihaa.jpg}
    \caption{Eeyore, or Ihaa, a very sad donkey.}
    \label{fig:eeyore}
  \end{center}
\end{figure}

You can create PDF files out of practically anything.
In Windows, you can download PrimoPDF or CutePDF (or some such) and install a
printing driver so that you can print directly to PDF files from any
application. There are also tools that allow you to upload documents in common
file formats and convert them to the PDF format.
If you have PS or EPS files, you can use the tools \texttt{ps2pdf} or
\texttt{epspdf} to convert your PS and EPS files to PDF\@.

% Comment: If your sentence ends in a capital letter, like here, you should
% write \@ before the period; otherwise LaTeX will assume that this is not
% really an end of the sentence and will not put a large enough space after the
% period. That is, LaTeX assumes that you are (for example), enumerating using
% capital roman numerals, like I. do something, II. do something else. In this
% case, the periods do not end the sentence.

% Similarly, if you do need a normal space after a period (instead of
% the longer sentence separator), use \  (backslash and space) after the
% period. Like so: a.\ first item, b.\ second item.

Furthermore, most newer editor programs allow you to save directly to the PDF
format. For vector editing, you could try Inkscape, which is a new open source
WYSIWYG vector editor that allows you to save directly to PDF\@.
For graphs, either export/print your graphs from OpenOffice Calc/Microsoft
Excel to PDF format, and then add them; or use \texttt{gnuplot}, which can
create PDF files directly (at least the new versions can).
The terminal type is \emph{pdf}, so the first line of your plot file should be
something like \texttt{set term pdf \ldots}.

To get the most professional-looking graphics, you can encode them using the
TikZ package (TikZ is a frontend for the PGF graphics formatting system).
You can create practically any kind of technical images with TikZ, but it has a
rather steep learning curve. Locate the manual (\texttt{pgfmanual.pdf}) from
your \LaTeX\ distribution and check it out. An example of TikZ-generated
graphics is shown in Figure~\ref{fig:page-merge}.

\begin{figure}[ht]
  \begin{center}
    \input{example_page-merge.tex}
    \caption{Example of a multiversion database page merge. This figure has
    been taken from the PhD thesis of Haapasalo~\cite{HaapasaloThesis}.}
    \label{fig:page-merge}
  \end{center}
\end{figure}

Another example of graphics created with TikZ is shown in
Figure~\ref{fig:tikz-examples}.
These show how graphs can be drawn and labeled.
You can consult the example images and the PGF manual for more examples of what
kinds figures you can draw with TikZ.

% These definitions are only used in the example images; you will not
% need them for your thesis...
\newlength{\graphdotsize}
\setlength{\graphdotsize}{1.7pt}
\newlength{\graphgridsize}
\setlength{\graphgridsize}{1.2em}
\begin{figure}[ht]
\begin{center}
\subfigure[Examples of obstruction graphs for the Ferry Problem]{
  \input{example_obstruction-grouped.tex}
}
\subfigure[Examples of star graphs]{
  \input{example_general-star-graphs.tex}
}
\caption{Examples of graphs draw with TikZ. These figures have been taken from a
course report for the graph theory course~\cite{FerryProblem}.}
\label{fig:tikz-examples}
\end{center}
\end{figure}



% \input{4methods.tex}

%\chapter{Methods}
%\label{chapter:methods}

You have now stated your problem, and you are ready to do something
about it!  \emph{How} are you going to do that? What methods do you
use?  You also need to review existing literature to justify your
choices, meaning that why you have chosen the method to be applied in
your work.

% An example of a traditional LaTeX table
% ------------------------------------------------------------------
% A note on underfull/overfull table cells and tables:
% ------------------------------------------------------------------
% In professional typography, the width of the text in a page is always a lot
% less than the width of the page. If you are accustomed to the (too wide) text
% areas used in Microsoft Word's standard documents, the width of the text in
% this thesis layout may suprise you. However, text in a book needs wide
% margins. Narrow text is easier to read and looks nicer. Longer lines are
% hard to read, because the start of the next line is harder to locate when
% moving from line to the next.
% However, tables that are in the middle of the text often would require a wider
% area. By default, LaTeX will complain if you create too wide tables with
% ``overfull'' error messages, and the table will not be positioned properly
% (not centered). If at all possible, try to make the table narrow enough so
% that it fits to the same space as the text (total width = \textwidth).
% If you do need more space, you can either
% 1) ignore the LaTeX warnings
% 2) use the textpos-package to manually position the table (read the package
%    documentation)
% 3) if you have the table as a PDF document (of correct size, A4), you can use
%    the pdfpages package to include the page. This overrides the margin
%    settings for this page and LaTeX will not complain.
% ------------------------------------------------------------------
% Another note:
% ------------------------------------------------------------------
% If your table fits to \textwidth, but the cells are so narrow that the text
% in p{..}-formatted cells does not flow nicely (you get underfull warnings
% because LaTeX tries to justify the text in the cells) you can manually set
% the text to unjustified by using the \raggedright command for each cell
% that you do not want to be justified (see the example below). \raggedleft
% is also possible, of course...
% ------------------------------------------------------------------
% If you need to have linefeeds (\\) inside a cell, you must create a new
% paragraph-formatting environment inside the cell. Most common ones are
% the minipage-environment and the \parbox command (see LaTeX documentation
% for details; or just google for ``LaTeX minipage'' and ``LaTeX parbox'').
\begin{table}
\begin{tabular}{|p{2cm}|p{3.8cm}|p{4.5cm}|p{1.1cm}|}
% Alignment of sells: l=left, c=center, r=right.
% If you want wrapping lines, use p{width} exact cell widths.
% If you want vertical lines between columns, write | above between the letters
% Horizontal lines are generated with the \hline command:
\hline % The line on top of the table
\textbf{Code} & \textbf{Name} & \textbf{Methods} & \textbf{Area} \\
\hline
% Place a & between the columns
% In the end of the line, use two backslashes \\ to break the line,
% then place a \hline to make a horizontal line below the row
T-110.6130 & Systems Engineering for Data Communications
    Software & \raggedright Computer simulations, mathematical modeling,
  experimental research, data analysis, and network service business
  research methods, (agile method) & T-110 \\
\hline
\multicolumn{2}{|p{6.25cm}|}{Mat-2.3170 Simulation (here is an example of
 multicolumn for tables)}& Details of how to build simulations & T-110 \\
% The multicolumn command takes the following 3 arguments:
% the number of cells to merge, the cell formatting for the new cell, and the
% contents of the cell
\hline
S-38.3184 & Network Traffic Measurements and Analysis
& \raggedright How to measure and analyse network
  traffic & T-110 \\ \hline
\end{tabular} % for really simple tables, you can just use tabular
% You can place the caption either below (like here) or above the table
\caption{Research methodology courses}
% Place the label just after the caption to make the link work
\label{table:courses}
\end{table} % table makes a floating object with a title

If you have not yet done any (real) metholodogical courses (but chosen
introduction courses of different areas that are listed in the
methodological courses list), now is the time to do so or at least
check through material of suitable methodological courses. Good
methodologial courses that consentrates especially to methods are
presented in Table~\ref{table:courses}. Remember to explain the
content of the tables (as with figures). In the table, the last column
gives the research area where the methods are often used. Here we used
table to give an example of tables. Abbreviations and Acronyms is also
a long table. The difference is that longtables can continue to next
page.



% \input{5implementation.tex}

%\chapter{Implementation}
%\label{chapter:implementation}

You have now explained how you are going to tackle your problem.
Go do that now! Come back when the problem is solved!

Now, how did you solve the problem?
Explain how you implemented your solution, be it a software component, a
custom-made FPGA, a fried jelly bean, or whatever.
Describe the problems you encountered with your implementation work.



% \input{6evaluation.tex}

%\chapter{Evaluation}
%\label{chapter:evaluation}

You have done your work, but that's\footnote{By the way, do \emph{not} use
shorthands like this in your text! It is not professional! Always write out all
the words: ``that is''.} not enough.

You also need to evaluate how well your implementation works.  The
nature of the evaluation depends on your problem, your method, and
your implementation that are all described in the thesis before this
chapter.  If you have created a program for exact-text matching, then
you measure how long it takes for your implementation to search for
different patterns, and compare it against the implementation that was
used before.  If you have designed a process for managing software
projects, you perhaps interview people working with a waterfall-style
management process, have them adapt your management process, and
interview them again after they have worked with your process for some
time. See what's changed.

The important thing is that you can evaluate your success somehow.
Remember that you do not have to succeed in making something spectacular; a
total implementation failure may still give grounds for a very good master's
thesis---if you can analyze what went wrong and what should have been done.




% \input{7discussion.tex}

%\chapter{Discussion}
%\label{chapter:discussion}

At this point, you will have some insightful thoughts on your implementation
and you may have ideas on what could be done in the future.
This chapter is a good place to discuss your thesis as a whole and to show your
professor that you have really understood some non-trivial aspects of the
methods you used\ldots



% \input{8conclusions.tex}

%\chapter{Conclusions}
%\label{chapter:conclusions}

Time to wrap it up!
Write down the most important findings from your work.
Like the introduction, this chapter is not very long.
Two to four pages might be a good limit.



% Load the bibliographic references
% ------------------------------------------------------------------
% You can use several .bib files:
% \bibliography{thesis_sources,ietf_sources}
\clearpage
\phantomsection
\addcontentsline{toc}{chapter}{Bibliography}
\bibliography{ref}


% Appendices go here
% ------------------------------------------------------------------
% If you do not have appendices, comment out the following lines
\appendix
% \input{appendices.tex}

\chapter{First appendix}
\label{chapter:first-appendix}

This is the first appendix. You could put some test images or verbose data in an
appendix, if there is too much data to fit in the actual text nicely.

For now, the Aalto logo variants are shown in Figure~\ref{fig:aaltologo}.

\begin{figure}
\begin{center}
\subfigure[In English]{\includegraphics[width=.8\textwidth]{aalto-logo-en}}
\subfigure[Suomeksi]{\includegraphics[width=.8\textwidth]{aalto-logo-fi}}
\subfigure[Pä svenska]{\includegraphics[width=.8\textwidth]{aalto-logo-se}}
\caption{Aalto logo variants}
\label{fig:aaltologo}
\end{center}
\end{figure}


% End of document!
% ------------------------------------------------------------------
% The LastPage package automatically places a label on the last page.
% That works better than placing a label here manually, because the
% label might not go to the actual last page, if LaTeX needs to place
% floats (that is, figures, tables, and such) to the end of the
% document.
\end{document}
